% last updated in April 2002 by Antje Endemann
% Based on CVPR 07 and LNCS, with modifications by DAF, AZ and elle, 2008 and AA, 2010, and CC, 2011; TT, 2014; AAS, 2016

\documentclass[runningheads]{llncs}
\usepackage{graphicx, hyperref}
\usepackage{amsmath,amssymb} % define this before the line numbering.
\usepackage{ruler}
\usepackage{color}
\usepackage[width=122mm,left=12mm,paperwidth=146mm,height=193mm,top=12mm,paperheight=217mm]{geometry}

\graphicspath{{figs/}}

\newcommand{\red}[1]{\textcolor{red}{#1}}

\begin{document}
% \renewcommand\thelinenumber{\color[rgb]{0.2,0.5,0.8}\normalfont\sffamily\scriptsize\arabic{linenumber}\color[rgb]{0,0,0}}
% \renewcommand\makeLineNumber {\hss\thelinenumber\ \hspace{6mm} \rlap{\hskip\textwidth\ \hspace{6.5mm}\thelinenumber}}
% \linenumbers
\pagestyle{headings}
\mainmatter
\def\ECCV18SubNumber{***}  % Insert your submission number here

\title{ClassificationCam: How can optics be used in convolutional neural networks? (not final)} % Replace with your title

\titlerunning{ECCV-18 submission ID \ECCV18SubNumber}

\authorrunning{ECCV-18 submission ID \ECCV18SubNumber}

\author{Anonymous ECCV submission}
\institute{Paper ID \ECCV18SubNumber}


\maketitle

\begin{abstract}
The abstract should summarize the contents of the paper. LNCS guidelines
indicate it should be at least 70 and at most 150 words. It should be set in 9-point
font size and should be inset 1.0~cm from the right and left margins. Paper page limit is 14 pages, excluding references.

\keywords{We would like to encourage you to list your keywords within
the abstract section}
\end{abstract}


\section{Introduction}
\label{sec:intro}
Deep neural networks have found success in a wide variety of applications, ranging from computer vision to natural language processing to game playing \cite{lecun2015deep}. Convolutional neural networks (CNNs), capitalizing on the spatial invariance of certain properties of images, have been especially popular in computer vision problems such as image classification, image segmentation, and even image generation \cite{krizhevsky2012imagenet,goodfellow2014generative,long2015fully}. As performance on a breadth of tasks has improved to a remarkable level, the number of parameters and connections in these networks has grown dramatically, and the power and memory requirements to train and use these networks have increased correspondingly. 

While the training phase of learning parameter weights is often considered the slow stage, large models also demand significant energy during inference due to millions of repeated memory references and matrix multiplications. For example, the final version of Google DeepMind's AlphaGo in \cite{silver2016mastering} used 40 search threads, 48 CPUs, and 8 GPU to play a game of Go. Live imaging and sensing applications face the additional challenge of power-hungry sensors and high bandwidth transfer of data to feed into the downstream computer vision algorithms \cite{likamwa2013energy}. For these reasons, it remains difficult for embedded systems such as mobile vision, autonomous vehicles and robots, and wireless smart sensors to deploy CNNs due to stringent constraints on power and bandwidth. 

Optical computing has been tantalizing for its high bandwidth and inherently parallel processing, potentially at the speed of light. Furthermore, certain linear transformations can be performed in free-space or on a photonic chip with minimal to no power consumption, e.g. a lens can take a Fourier transform ''for free`` \cite{yang2013chip,goodman2008introduction}. Nonlinear operations could also be addressed optically, drawing on passive nonlinear materials or devices whose refractive indices or transmission states are dependent on optical input \cite{gibbs2012optical,christodoulides2010nonlinear}. An optimizable and scalable set of optical configurations that preserves these advantages and serves as a framework for building optical CNNs would be of interest to computer vision, robotics, machine learning, and optics communities. Optical implementation could also have the potential to expand beyond traditional operations of CNNs, potentially by harnessing wave optics and quantum optics in new ways. 

We take initial steps toward this broader goal from a computational imaging approach, integrating image acquisition with computation via co-design of optics and algorithms. By pushing one or more layers of a CNN into the optics, we can reduce the workload of the electronic processor when performing inference with a CNN. Imaging systems are often characterized by their point spread function (PSF), which describes how a single point source of light propagates through the system. Hence, for a simple linear and space-invariant system, the image recorded at the output is the convolution of the original object with the system PSF \cite{goodman2008introduction}. This built-in convolution motivated us to explore how we could use optics to replace one or more of the layers in a CNN. 

In this paper, we propose a toolbox of optical building blocks that could be used to implement common neural network layers. To evaluate these components, we build a simulation framework for testing a few variations of optical CNNs with the relevant physical constraints, including learned optical correlators, hybrid optoelectronic CNNs, and fully optical CNNs. We train these networks to perform image classification on a few different datasets (MNIST, GoogleQuickdraw, or CIFAR-10), and we compare the simulated ONN accuracy against the unconstrained computer implementation of the same network structure. To demonstrate the validity of our simulations, we build a hybrid optoelectronic two-layer network with an optical convolutional layer and electronic fully connected layer for CIFAR-10 classification. We compare performance with the same inference performed on the computer, with and without the simulated physical constraints of an optical setup. 

\textit{Overview of limitations.} 
While the proposed ONN architectures offer lower power inference on classification tasks, the physical image formation imposes several constraints on the CNN architecture, including nonnegative signal and weights when using incoherent light, no bias, limited set of nonlinearities, etc. We will discuss in more detail in the paper how much each of these constraints limit the performance of our system. Here we demonstrate proof-of-concept with bulk optics and free-space propagation, which is not necessarily practical or scalable to commercial applications. However, photonic integrated circuits could significantly help in both these regards \cite{sun2013large,rechtsman2013photonic,shen2017deep}. Combination of these next-generation large-scale photonic circuits with compressed deep learning models could provide a potential route for high performance ONNs.


\section{Related Work}
\label{sec:related}
\paragraph{Convolutional neural networks.} 
Artificial neural networks were proposed in X. Early networks were composed of fully connected layers with nonlinear activation functions in between, inspired by the canonical biological neuron and its thresholded activation. Convolutional layers were popularized by LeCun and ... in image classification CITE. Convolutional layers allow for weight sharing... Since then, deeper, more complex, etc. 

As embedded vision and even continuous mobile vision become 
hardware
incorporation of image processing on the sensor chip, eliminating or reducing the need to shuttle full image data to a processor. These chips have been designed to detect edges and orientations and to perform wavelet or discrete cosine transforms \cite{gruev2002implementation} [RedEye]. Most of these approaches still rely on electronic computation on the image sensor chip, whereas our goal is all-optical implementation with no additional power input. 

\paragraph{CNN architecture variations}

Our goal is to match performance with a constrained optical setup, so also relevant to highlight are CNNs with non-standard architectures that may align with physical designs. Omission of fully connected layers, i.e. fully convolutional with global average pooling at the top layer has proven to be successful in \cite{lin2013network,iandola2016squeezenet}. Analysis of CNN operations in the Fourier domain, introducing spectral pooling and regularization \cite{rippel2015spectral}. Relevant because we can also access optical Fourier plane. We also note the work in the complex-valued deep neural networks \cite{trabelsi2017deep}, as coherent optical signals may be an effective means of propagating complex-valued data.
 

\paragraph{Optical computing and computational light transport.}
In the computational imaging community, many new optical system designs exploit the physical propagation of light to 
The co-design of optics and algorithms 

Optical computing offers high bandwidth, but high cost. Optoelectronics and fully optical. Optical solutions to NP-complete problems that are faster than electronic computation \cite{wu2014optical}.

% Computational photography has some intersection with optical computing in that they may perform some operations on the input signal optically, but they are also distinct in that they work with spatially organized inputs that come from physical world (incoherent light). Coded apertures and PSF engineering can perform filtering [CITE]. Optical correlators that essentially perform template matching on images have been explored for optical target detection and tracking \cite{manzur2012optical, javidi1995optical}.  

\paragraph{Optical neural networks.}   The concept of an optical neural network (ONN) captured the attention of many in the late 1980s to mid-1990s, primarily due to the capability of optics to perform the expensive matrix multiply of a fully connected layer. In 1985, an optoelectronic implementation of the Hopfield model, a basic model of a recurrent neural network, was created with one-dimensional (1D) LED array input signals and a binary transmission mask \cite{farhat1985optical}. This model divided the weight matrix into two parts, positive and negative, and required electronics for subtraction of the two parts and signal thresholding. Psaltis et al. further explored the potential of dynamic photorefractive crystals to store neural network weights, which could allow for optical backpropagation-based learning in ONNs \cite{psaltis1988adaptive}. Meanwhile, the optoelectronic network of a Hopfield model was extended to 2D signals by partitioning the pixels of a liquid crystal television to store an array of smaller 2D patterns \cite{lu1989two}. Furthermore, an optical thresholding perceptron was implemented with liquid crystal light valves (LCLV), which disposed of the need to convert between optical and electronic signals between layers \cite{saxena1995adaptive}. A more extensive overview of the varied implementations of ONNs can be found in \cite{denz2013optical}.

Despite the accumulation of insights in this area, as neural networks fell out of the spotlight, the demand for ONNs also waned. However, with the resurgence of CNNs that are far more powerful and computationally expensive than before, there is renewed interest in optical computing \footnote{Fathom Computing (\url{fathomcomputing.com}), Lightelligence (\url{lightelligence.ai}), Optalysis (\url{optalysys.com)}}. Recent works that connect efforts of the last century to modern hardware include a two-layer fully connected neural network based on programmable photonic circuits \cite{shen2017deep} and  a recurrent neural network with DMD-based weights \cite{bueno2017reinforcement}. However, none of the ONNs mentioned previously involve convolutional layers, which have become essential in computer vision applications. The ASP Vision system approaches the task of designing a hybrid optoelectronic CNN, using angle sensitive pixels to approximate the first convolutional layer of a typical CNN, but it is limited to a fixed set of convolution kernels \cite{chen2016asp}. Our goal is to design a system with optimizable optical elements to demonstrate low-power inference by a custom optical or optoelectronic CNN.




\section{ONN Toolbox, pp. 5-7}
\label{sec:toolbox}
In this section we present building blocks for an optically implemented CNN. In this rest of this paper we focus on the convolutional layer, but we will briefly discuss other layers here too.
\begin{itemize}
\item Convolutional layer:
	Small kernel, tiled kernels, single large kernel (\red{Fig. 1})
\item Other layers that we didn’t test in detail: \\
Optical nonlinearities \\ 
	Max/avg pooling with spectral pooling \\ 
	Fully connected layers	
\end{itemize}

% \paragraph{CNN architecture variations} Our goal is to match performance with a constrained optical setup, so also relevant to highlight are CNNs with non-standard architectures that may align with physical designs. Omission of fully connected layers, i.e. fully convolutional with global average pooling at the top layer has proven to be successful in \cite{lin2013network,iandola2016squeezenet}. Analysis of CNN operations in the Fourier domain, introducing spectral pooling and regularization \cite{rippel2015spectral}. Relevant because we can also access optical Fourier plane. We also note the work in the complex-valued deep neural networks \cite{trabelsi2017deep}, as coherent optical signals may be an effective means of propagating complex-valued data.

\section{Training, pp. 7}
Here we talk about how we train the ONN offline in Tensorflow. PSF optimization followed by phase mask optimization OR end-to-end phase mask optimization.

\section{Simulations, pp. 8-10}
\label{sec:simulation}
We use simulations to better understand the performance of optical CNNs. \\
\red{Fig. 2}: diagram of possible ONN models.
\red{Table 1:} Results.
\begin{itemize}
\item Introduce toy classification problem(s?), discuss constraints.
\item Learned optical correlator (single optical conv. layer) \\ Here we are able to use end-to-end learning. \\
\red{Possible figure with learned phase mask and PSF.}
\item Hybrid optoelectronic (one optical conv. layer)\\ 
	Talk about the dual channel positive and negative weights\\
	Grayscale \\
	With color filters – \red{Vincent?}\\
\red{Fig. 3: Hybrid ONN phase masks and PSFs.}
\item All-optical convolutional neural network \\
Doesn’t fully work, but can discuss some results
\end{itemize}

\section{Optical Prototype, pp. 10-11}
\label{sec:prototype}
Implement the hybrid optoelectronic two-layer neural network. Goal is to show that the hybrid ONN can perform on par with the electronic ONN, with the same number of layers, and better than the electronic ONN with one fewer layer.\\
\red{Fig. 4: optical setup} \\
\red{Fig. 5: actual PSF and sample images} \\
\red{Table. 2: results}

\section{Discussion, pp. 12-13}
\label{sec:discussion}
\begin{itemize}
\item Not straightforward to generalize first optical conv. layer to multiple optical layers
\item Discuss importance of negative weights – \red{Vincent?}
\item Instead of trying to replicate a CNN exactly, could take advantage of optical transformations that aren't as practical in computations. For example, we use a 4f system for convolution, but this requires two extra lenses. Perhaps a single custom learned optical element can be used instead.
\item In the future, exploit other properties of light (polarization, phase)	
\item Specifically, coherent light and holography, photonics
\end{itemize}

\section{Conclusion, p. 14}
\label{sec:conclusion}
Important step towards optical CNNs. We hope this will inspire more research in the area. 

\clearpage

\bibliographystyle{splncs}
\bibliography{../bibliography}
\end{document}
