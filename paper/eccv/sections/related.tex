\paragraph{CNNs and architecture variations.} Artificial neural networks were proposed in X. Early networks were composed of fully connected layers with nonlinear activation functions in between, inspired by the canonical biological neuron and its thresholded activation. Convolutional layers were popularized by LeCun and ... in image classification CITE. Convolutional layers allow for weight sharing... Since then, deeper, more complex, etc. Since an optical implementation of a CNN comes with certain constraints and challenges, we wanted to see what types of CNNs have been explored with non-standard architectures that may align with physical designs. Omission of fully connected layers, i.e. fully convolutional with global average pooling at the top layer has proven to be successful in \cite{lin2013network,iandola2016squeezenet}. Analysis of CNN operations in the Fourier domain, introducing spectral pooling and regularization \cite{rippel2015spectral}. Relevant because we can also access optical Fourier plane. We also note the work in the complex-valued deep neural networks \cite{trabelsi2017deep}, as coherent optical signals may be an effective means of propagating complex-valued data.

\paragraph{Optical computing} High bandwidth, but high cost. Optoelectronics and fully optical. Optical solutions to NP-complete problems that are faster than electronic computation \cite{wu2014optical}.  In the early days of CNNs, there was also momentum for optical implementation, optical neural networks (ONNs). Adaptive optical network using volume holographic interconnects in photorefractive crystals \cite{psaltis1988adaptive}. Hybrid optoelectronic network with feedback loop, computer for subtraction and thresholding operations \cite{lu1989two}. Optical thresholding perceptron implemented with liquid crystal light valves (LCLV) \cite{saxena1995adaptive}. There has also been much development in photonic computing. Recently, two-layer fully connected NN demonstrated with intermediate simulated nonlinearity units on 1D data \cite{shen2017deep}. However, this required photodetection and reinjection, and it did not involve convolutional layers. Optalysys? We do not work with photonic circuits here, but we think they may be worth exploring for larger networks.

\paragraph{Computational cameras.} Computational photography has some intersection with optical computing in that they may perform some operations on the input signal optically, but they are also distinct in that they work with spatially organized inputs that come from physical world (incoherent light). Coded apertures and PSF engineering can perform filtering [CITE]. Optical correlators that essentially perform template matching on images have been explored for optical target detection and tracking \cite{manzur2012optical, javidi1995optical}. Somewhat similar to our goal is focal plane processing, which refers to the incorporation of image processing on the sensor chip, eliminating or reducing the need to shuttle full image data to a processor. These chips have been designed to detect edges and orientations and to perform wavelet or discrete cosine transforms \cite{gruev2002implementation} [RedEye]. Most of these approaches still rely on electronic computation on the image sensor chip, whereas our goal is all-optical implementation with no additional power input. Chen et al. use optically designed angle sensitive pixels, photodiodes with integrated diffraction gratings producing Gabor wavelet impulse responses, to approximate the kernels of the first layer of a typical convolutional neural network \cite{chen2016asp}. However, this design is limited to a fixed set of convolution kernels, and the output still has to be shuttled to a computer for further processing. Our goal is to build an end-to-end classification system with flexible and rearrangeable optical units that allows for custom optical CNNs.